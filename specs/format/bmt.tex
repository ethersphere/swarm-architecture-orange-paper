The hashing method used to obtain the address of the default content addressed chunk is called the \gloss{binary Merkle tree hash}, or \emph{BMT Hash} for short. 

\subsubsection{Calculating the BMT hash}

The base segments of the binary tree are subsequences of the chunk content data. 
The size of segments is 32  bytes, which is the digest size of the \emph{base hash} used to construct the tree. 
Given the Swarm hash tree used to represent files (see \ref{spec:format:files}) assumes that intermediate chunks package references to other chunks. 

Obtaining the BMT hash of a sequence involves the following steps:

\begin{enumerate}
\item \emph{padding} - If the content is shorter than the maximum chunk Size  (4096 bytes, \ref{sec:content-addressed-chunks}), it is padded with zeros up to chunk size. Note that this zero padding is only for hashing and does not impact chunk data sizes.
\item \emph{chunk data layer} - Calculate the base hash of \emph{pairs of segments} in the padded chunk, i.e., segment size ($2 * 32$) units of data and concatenate the results.
\item \emph{building the tree} - Repeat previous step on the result until the result is just one section
\item \emph{calculate span} - Calculate the span of the data, i.e., the size of the data that is subsumed under the chunk represented by the unpadded data as a 64-bit little-endian integer value (see  \ref{spec:format:files})            
\item \emph{integrity protection} - Prepend the span to the root hash of the binary tree and calculate the \emph{base hash} of the data
\end{enumerate}

\begin{definition}[BMT hash]\label{def:bmt-hash}
\begin{lstlisting}[language=buzz1]
// /bmt

define function hash payload 
    with span
as
    @padded = @payload as [:chunk size]byte    // use zero padding 
    // for BMT hashing only
    hash @span and root of @padded over chunk size 
    
define function root of @section []byte
    over @len uint
as
    return hash @section        // data level
        if @len == 2 * segment size
    @len /= 2                  // recursive call
    @children = @section each @len go self over @len
    wait for @children 
        join hash
    

\end{lstlisting}
\end{definition}

\subsubsection{Inclusion proofs}

Having the segments align with the hashes packaged in these chunks one can extend the notion of inclusion proofs to files.
The BMT hash enables compact 3rd party verifiable segment inclusion proofs.

\red{to be specified}