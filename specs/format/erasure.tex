
Erasure coding interface provides wrappers \lstinline{extend/repair} for the encoder/decoder that work directly on a list of chunks.%
%
\footnote{Cauchy-Reed-Solomon erasure codes based on \url{https://github.com/klauspost/reedsolomon}.
}

Assuming $n$ out of $m$ coding.
\lstinline{extend} takes a list of $n$ data chunks and an argument for the number of required parities. It returns the parity chunks only.
\lstinline{repair} takes a list of $m$ chunks (extended with \emph{all} parities) and an argument for the number of parities $p=m-n$, that designate the last $p$ chunks as parity chunks. It returns the list of $n$ repaired data chunks only.
The encoder does not know which parts are invalid, so missing or invalid chunks should be set to \lstinline{nil} in the argument to repair.
If parity chunks are needed to be repaired, you call \lstinline{repair @chunks with @parities; extend with @parities}

\begin{definition}{CRS erasure code interface definition}\label{def:crs}
\begin{lstlisting}[language=buzz1]

define function extend []chunk as @chunks
    with @parities uint
    return [@parities]chunk

define function repair []chunk as @chunks  
    with @parities uint
   return [@chunks length - @parities]chunk

\end{lstlisting}
\end{definition}
