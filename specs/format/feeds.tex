
Extract the fields out of the JSON and build a query string as below:

`POST /bzz-resource:/?topic=<TOPIC>&user=<USER>&level=<LEVEL>&time=<TIME>&signature=<SIGNATURE>`

body: binary stream with the update data.

### To get the last update:

* `GET /bzz-resource:/?topic=<TOPIC>&user=<USER>`
* `GET /bzz-resource:/<MANIFEST OR ENS NAME>`

> Note:
> * Again, if `topic` is omitted, it is assumed to be zero, `0x000...`
> * if `name=<name>` is provided, a subtopic is composed with that name
> 
> A common use is to omit topic and just use `name`, allowing for human-readable topics.
> Thus, this is also valid: `GET /bzz-resource:/?name=profile-picture&user=<USER>`

### To get a previous update:

Add an addtional `time` parameter. The last update before that time will be looked up.

* `GET /bzz-resource:/?topic=<TOPIC>&user=<USER>&time=<T>`
* `GET /bzz-resource:/<MANIFEST OR ENS NAME>?time=<T>`

* `hint.time`: Time at when you think the last update happened
* `hint.level`: Integer. Approximate period you think the updates where happening at, expressed as `log2(T)`, rounded up. For example, a resource updating every 300 seconds, level should be set to `9`. `log2(300) = 8.22`. See the Adaptive Frequency algorithm below for details on this.

Note that this would only affect first lookups. Your swarm node will keep track of last updates and automatically use the last seen update as a hint. Using these parameters would override that automatic hint.

### To publish a manifest:

`POST /bzz-resource:/?topic=<TOPIC>&user=<USER>&manifest=1` with an empty body.

Note: this functionality could be moved to the client and removed from the node, since this just creates a JSON and publishes it to `bzz-raw`, so the client could actually create this itself and call `client.UploadRaw()`. Don't expect this call to be available in future releases.


## CLI

### Creating a resource manifest:

`swarm resource create` is redefined as a command line to create and publish a MRU manifest only.

`swarm resource create [command options]`
```
creates and publishes a new Mutable Resource manifest pointing to a specified user's updates about a particular topic.
          The topic can be specified directly with the --topic flag as an hex string
          If no topic is specified, the default topic (zero) will be used
          The --name flag can be used to specify subtopics with a specific name
          The --user flag allows to have this manifest refer to a user other than yourself. If not specified,
          it will then default to your local account (--bzzaccount)

OPTIONS:
--name value   User-defined name for the new resource, limited to 32 characters. If combined with topic, the resource will be a subtopic with this name
--topic value  User-defined topic this resource is tracking, hex encoded. Limited to 64 hexadecimal characters
--user value   Indicates the user who updates the resource
```

### Update a resource

`swarm resource update [command options] <0x Hex data>`
```
creates a new update on the specified topic
          The topic can be specified directly with the --topic flag as an hex string
          If no topic is specified, the default topic (zero) will be used
          The --name flag can be used to specify subtopics with a specific name.
          If you have a manifest, you can specify it with --manifest instead of --topic / --name
          to refer to the resource

OPTIONS:
--manifest value  Refers to the resource through a manifest
--name value      User-defined name for the new resource, limited to 32 characters. If combined with topic, the resource will be a subtopic with this name
--topic value     User-defined topic this resource is tracking, hex encoded. Limited to 64 hexadecimal characters
```

### Quick and dirty test:

In the example, the user wants to publish his/her profile picture so it can be found by anyone who knows his/her Ethereum address.

```bash
# (OPTIONAL) Publish a manifest:
swarm --bzzaccount "your public key" resource create --name "profile-picture"

# the above command will output a manifest hash. We will refer to it as $MH later on

# Publish the first update:
$ IMAGE=$(swarm up myprofilepicture.jpg) && swarm --bzzaccount "<YOUR ADDRESS>" resource update --name "profile-picture" "0x1b20$IMAGE"

# To retrieve the latest update:
$ curl 'http://localhost:8500/bzz-resource:/?name=profile-picture&user=<YOUR ADDRESS>'

# Alternatively, if we created a manifest, we can use it:
$ IMAGE=$(swarm up myprofilepicture.jpg) && swarm --bzzaccount "your public key" resource update --manifest "$MH" "0x1b20$IMAGE"

# Your last profile picture can be viewed in:
# http://localhost:8500/bzz:/$MH
# (Only if a manifest was published)
```
