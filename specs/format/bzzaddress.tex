\subsection{Overlay address \statusyellow}

Swarm's overlay network uses 32-byte addresses. In order to help  uniform utilisation of the address space,  these addresses must be derived using a hash function. A Swarm node must be associated with an Ethereum account called the \gloss{Swarm base account} or \gloss{bzz account}.
The node's overlay address is derived as hash of the corresponding  public key.
 of the 

\begin{definition}{Swarm overlay address of node A}\label{def:overlay}
\begin{equation}
\mathit{overlayAddress}(A) \defeq \mathit{Hash}(\mathit{ethAddress}|\mathit{bzzNetworkID})         
\end{equation}
where
\begin{itemize}
\item $\mathit{Hash}$ is the 256-bit Keccak SHA3 hash function
\item \emph{ethAddress} - the ethereum address  (bytes,  not hex) derived from the node's base account public key: $\mathit{account}\defeq\mathit{PubKey}(K_A^\mathit{bzz})[12:32]$), where
    \begin{itemize}
    \item \emph{PubKey} is the \emph{uncompressed} form of the public key of a keypair \emph{including} its $04$ (uncompressed) prefix.
    \item $K_A^\mathit{bzz}$ refers to the node's bzz account key pair
    \end{itemize}
\item \emph{bzzNetworkID} is the bzz network id of the swarm network serialised as a little-endian binary \emph{uint64}.
\end{itemize}
\end{definition}

\subsection{Underlay address \statusyellow}

To enable peers to locate the a node on the network, the overlay address is paired with an underlay address. The underlay address is a string representation of the node's network location on the underlying transport layer. 

\begin{definition}{underlay}\label{def:underlay}
\begin{lstlisting}[]
// ID: /swarm/handshake/1.0.0/
}
\end{lstlisting}
\end{definition}

% As long as swarm runs of devp2p, we use the \gloss{enode URL scheme} representation:

% \begin{definition}{devp2p underlay address of a node}\label{def:underlay}
% $\mathit{underlay}(n) \defeq $
% \begin{verbatim}
%     "enode://"<NODEID>"@"<HOST>(":"<TCPPORT>("?udp="<UDPPORT>))
% \end{verbatim}
% \end{definition}


\subsection{BZZ address \statusyellow}

Bzz address is functionally the pairing of overlay and underlay addresses. In order to ensure that an address is derived from an account the node possesses as well as verifiably attest to an underlay address a node can be called on, bzz addresses are communicated in the following transfer format:

\begin{definition}{Swarm bzz address transfer format} \label{def:bzz-address}
\begin{lstlisting}[]
// ID: /swarm/handshake/1.0.0/bzzaddress
// Serialization: Varint delimited Protobuf
syntax = "proto3";

message BzzAddress {
    bytes Underlay = 1;
    Signature Sig  = 2;
    bytes Overlay  = 3; 
}
\end{lstlisting}
\end{definition}

Here the signature is attesting to an underlay address for a network:

\begin{definition}{signed underlay address of node A}\label{def:signed-underlay}
\begin{equation}
\mathit{signedUnderlay}(A) \defeq \mathit{Sign}(\mathit{underlay}|\mathit{bzzNetworkID})         
\end{equation}
\end{definition}

of the underlay with bzz network ID appended as plaintext and hashes the resulting public key together with the bzz network ID. 

\begin{definition}{bzz types}\label{def:bzz-types}
\begin{lstlisting}[language=buzz1]
// /swarm/bzz

define type overlay as [segment.size]byte

define type account as crypto/keypair

define type eth.address of account 
as
    hash @account public key from 12

define function address of account 
    for uint64 as @network
as
    hash @account eth.address
        and @network 
            as overlay

\end{lstlisting}
\end{definition}

Note that the overlay address is not explicitly part of the construct. It is not needed since it can be calculated from signature, underlay and the network ID.



In order to get the overlay address from the transfer format peer info, one recovers from signature the peer's base account public key using the plaintext that is constructed as per \ref{def:signed-underlay}. From the public key, the overlay can be calculated as in \ref{def:overlay}.
