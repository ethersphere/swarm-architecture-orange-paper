Content can be pinned in two different ways. One is during the content upload and the other is thereafter.

    Pinning during upload

    Add a header “x-swarm-pin” and set it to “true” when uploading content. This will upload the file and then pin it too. This method can be used for Tar, Multipart and raw file uploads too.

curl -H "Content-Type: application/x-tar" -H "x-swarm-pin: true"  --data-binary @files.tar http://localhost:8500/bzz:/

    Pinning after upload

    If an already uploaded content needs to be pinned, the following HTTP API should be used.

to pin a Swarm collection
POST /bzz-pin:/<MANIFEST OR ENS NAME>

to pin a RAW file in Swarm
POST /bzz-pin:/<SWARM RAW FILE HASH>/?raw=true

Note

When pinning a already uploaded file, make sure that the entire file content is available locally by issuing a download once.
Unpinning Content

An already pinned file can be unpinned at anytime. Once the collection is unpinned, the contents will follow the FIFO rule and may be garbage collected in future.

DELETE /bzz-pin:/<MANIFEST OR ENS NAME OR SWARM RAW FILE HASH>

Listing Pinning Info

Pinned contents and their information can be viewed at any point using this API. Information includes The pinned hash, whether the pinned content is a collection or RAW file, the pinned content size in bytes and the no of time the content is pinned.

% \lstset{basicstyle=\footnotesize}
% \lstinline{GET /bzz-pin:/}

\begin{lstlisting}

GET /bzz-pin:/

[
 { "Address"    : "0x94f78a45c7897957809544aa6d68aa7ad35df695713895953b885aca274bd955",
   "IsRaw"      : "false",
   "FileSize"   : "12046",
   "PinCounter" : "2",
 },
 { "Address"    : "0xccef599d1a13bed9989e424011aed2c023fce25917864cd7de38a761567410b8",
   "IsRaw"      : "true",
   "FileSize"   : "146",
   "PinCounter" : "5",
 },
]

    
\end{lstlisting}
Note

The information will be returned in JSON format shown above
