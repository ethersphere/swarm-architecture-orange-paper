\section{Conventions}\label{spec:conventions}
\section{Notation}\label{spec:notation}

This document uses the \texttt{ABNF} convention for defining data types, as described in the IETF RFC 5234. \cite{IETF:ABNF}


The basic identifiers are defined as follows:

\begin{lstlisting}[numbers=none]
UINT64MAX	= 18446744073709551615
UINT32MAX	= 4294967295
UINT16MAX	= 65535
UINT8MAX	= 255
UINT64		= %d0-UINT64MAX
UINT32		= %d0-UINT32MAX
UINT16		= %d0-UINT16MAX
UINT8		= %d0-UINT8MAX

BOOL		= BIT

TIMESTAMP	= UINT32
\end{lstlisting}

The construct \texttt{LIST} explicitly defines a list of elements. Elements may be of any type, and are separated by spaces. This identifier has implications for the serialization of data. 

\section{SSerialisation} \label{sec:rlp}

Swarm uses devp2p/RLPX for underlay p2p transport (see \ref{sec:underlay-transport}) which in turn uses \gloss{RLP} \cite{ETHWIKI:RLP} for serialisation. 
In this document, protocol message payload data will be specified in ABNF format. We define the following mappings from ABNF data type definitions to RLP data types:

\begin{definition}{Mapping of ABNF data types to RLP data types}\label{sec:abnf-rlp}
\begin{description}
\item [BOOL] is encoded as an RLP encoded integer with length of one byte
\item [OCTET] is encoded as an RLP encoded integer with length of one byte
\item [TIMESTAMP] is encoded as an RLP encoded variable length integer
\item [\%x\#\#] hexadecimal literals are encoded as encoded integers with length of one byte, corresponding to its hexadecimal (not its string) value
\item [LIST] as defined in \ref{abnf-list-definition} will be RLP encoded as RLP lists.
\end{description}
\end{definition}

All other data is encoded as RLP lists.

The outermost element of a serialization is always a LIST item%
%
\footnote{This is due to the fact that the reference implementation is written in golang, and the default RLP deserialization of golang structs treats the struct itself as a list.}

For example, a message structure with only one field of \texttt{3OCTET} data will in reality be \texttt{LIST(3OCTET)}. If  the data is \texttt{0x666f6f} it will serialize to \texttt{0xc4 83 66 6f 6f}, which breaks down to:

\begin{lstlisting}[numbers=none]
LIST:	0xc4 (list, 4 bytes long)
3OCTET:	0x83 (string, 3 bytes long)
data: 	0x66 0x6f 0x6f
\end{lstlisting}
